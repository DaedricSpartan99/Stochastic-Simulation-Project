\section{Introduction}

This project is based on the paper "Multilevel Monte Carlo Methods" by Giles \cite{giles_2015} which explores the field of Monte Carlo methods and their application to numerical integration and optimization. Monte Carlo methods are a class of statistical techniques that use random sampling to approximate complex computations or estimations that are otherwise difficult to solve analytically. They are widely used in a variety of fields, including physics, finance, and engineering.

In this coursework, we will first introduce the concept of multi-level Monte Carlo (\textit{MLMC}) methods, which are a variant of traditional Monte Carlo methods that aim to reduce the computational cost of numerical integration by taking advantage of the hierarchical structure of the problem being solved. \textit{MLMC} methods work by dividing the problem into a hierarchy of levels, each of which can be solved independently using Monte Carlo techniques. This allows for a more efficient use of computational resources, as the number of samples required at each level can be adjusted based on the relative error and variance at that level.
Finally, \textit{MLMC} methods will be compared to the traditional Monte Carlo in terms of efficiency and \textit{mean square error} (MSE) reduction.