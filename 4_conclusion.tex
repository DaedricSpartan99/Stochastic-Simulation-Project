\section{Conclusion}
In conclusion, the use of MLMC for option pricing has been shown to be a powerful and efficient tool for solving high-dimensional and complex problems in comparison with classical Monte Carlo methods. By utilizing a multi-level approach, MLMC is able to significantly reduce the number of samples required to achieve a desired level of accuracy (MSE bound), compared to traditional MCMC methods.
However, the use of this method also comes with some limitations and challenges. The method assumes the underlying asset prices follow a geometric brownian motion, which might not be suitable in cases where the underlying asset prices exhibit different behavior. Furthermore, the method assumes the perfect knowledge of parameters, such as drift $r$ and volatility $\sigma$, which in this homework are known and it might not always be the case in the real world.
Overall, the use of MLMC method for option pricing can provide a significant advantage in terms of computational and memory efficiency, but it is important to consider the assumptions and limitations of the method before applying it to real-world necessities.




